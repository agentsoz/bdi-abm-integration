\documentclass{article} % ----- Lamport s 160 ff
\usepackage{a4}
\usepackage{fullpage}
%\special{header=draft.ps}
\usepackage{epsfig}
\usepackage{pdfpages}
\usepackage{url}
%\usepackage{longtable}
\usepackage[margin=0.7in]{geometry}
\usepackage{csquotes}
\usepackage{titlesec}
\usepackage{amssymb,amsmath}
\titleformat{\section}{\large\bfseries}{\thesection}{1em}{}

\newcommand{\Omit}[1]{}

\setlength{\parindent}{0pt}
\setlength{\parskip}{1.5ex plus 0.3ex}

%%%%%%%%%%%%%%%%%
% tighter lists
%%%%%%%%%%%%%%%%%
\newcounter{bean}

\newenvironment{tightenumerate}{
                \begin{list}{
                  {\mbox {
                      \arabic{bean}.\/}}}{\usecounter{bean}
                      \setlength{\itemsep}{-1pt}\setlength{\topsep}{0pt}}}{
                \end{list}}

\newenvironment{tightitemize}{
                \begin{list}{$\bullet$}{
                    \setlength{\itemsep}{-1pt}}{\setlength{\topsep}{0pt}}}{
                \end{list}}



\begin{document}
\begin{center}
{\Large \bf{Conservation Workshop Notes}}
\end{center}

These notes are the results of the workshop with Sarah, Sayed, Fiona,
Lin, Dhirendra, where we decided on some details of the
project. Produced months later from handwritten notes. It is
complemented by a section on the actual implementation.

We decided to model a situation where people's bidding behaviour is
influenced by a \textit{Conservation Ethic} (C) and a \textit{Profit
  Motive} (P).
These would both be represented by a measure/barometer, which is
affected (goes up/down) based on outcomes of auctions.

Agents can be split into 4 groups: HighC\_LowP, HighC\_HighP, LowC\_LowP,
LowC\_HighP. To avoid sharp behaviour changes around threshholds, for
each auction we categorise an agent as high on each measure, with a
probability that increases as they approach or exceed the threshold.

We describe the behaviour of these agents in terms of the
probability that they will bid at all, and the nature of the bids they
will submit. We also describe how successful participants,
unsuccessful participants and non-participants are affected by the auction
outcomes overall, in terms of adjustment to their C measure. All
agents adjust their P measure proportional to the amount of overall
profit in the winning bids, unless profit is low in which case there
is a small static increase. Cost recovery average outcomes have no
effect on P. 

\section{Behaviour of agent groups}
\subsection{HighC\_LowP}
Agents in this group are assumed to have a high probability of
participating in the auction, and are expected to place many bids that are
somewhere close to cost - can be slightly above or slightly below.

The conservation ethic of the three subgroups is affected as follows:
\begin{tightenumerate}
\item Successful participants:\\
if the profit is low to
medium, this will decrease their conservation ethic. If it is high it
will increase the conservation ethic. If it is cost recovery there
will be no effect on conservation ethic.
\item Unsuccessful participants:\\
conservation ethic will decrease proportional to the profit obtained by
others; cost recovery results in small decrease in conservation ethic.
\item Non-participants:\\
same as for unsuccessful participants.
\end{tightenumerate}

\subsection{HighC\_HighP}
Agents will have a high probability of participating, and will place a
moderate number of bids with moderate profit.

The conservation ethic is affected  in the same way as HighC\_LowP
(i.e. the key aspect is whether they currently have high or low
conservation ethic).

\subsection{LowC\_HighP}
Agents will participate according to the probability that there is a
chance of making a substantial profit, calculated by the percentage of
successful bids in the last three rounds which made a medium or high
profit. They will place only bids that can give a substantial profit
(calculated as a dollar amount, not as a percentage).

Conservation ethic is affected as follows:
\begin{tightenumerate}
\item Successful participants:\\
If profit is in the range 0--medium threshold, then conservation ethic
will decline; else conservation ethic will increase proportional to
profit. 
\item Unsuccessful participants:\\
Small fixed negative effect on conservation ethic.
\item Non-participants:\\
Any payment leads to a small decline in conservation ethic,
proportional to the profit.
\end{tightenumerate}

\subsection{LowC\_LowP}
Agents will participate with low probability, with a random number of
bids with medium profit.  Conservation ethic is affected in the same
way as for LowC\_HighP agents.

\section{Effect of social norms}
We assume that as norms change regarding conservation ethic, so does
the conservation ethic of individual agents. Thus at each step, an
average conservation ethic is calculated, and agents below this value
increase their conservation ethic, proportional to their distance from
the average.
{\it Lin says: maybe if the conservation ethic is trending upwards,
  agents above the average should also increase their C? If they see
  themselves as conservationists they will try to stay above the
  average? Or if their distance from the average has decreased since
  last step, then with some probability, restore the distance from the
  average? Need to discuss this...}

\section{Implementation}
\subsection{Assigning high vs low C and P groups at each auction}
Both conservation ethic barometer and profit motive barometer can vary 
from 0 to 100. The upper limit of both of these barometers are configurable 
and currently we have configured them to 100. The percentage of agents who 
have high conservation ethic barometer (highCEAgentsPercentage) ranges 
from 0-100\%. The number of landholders is configurable and currently we 
have set it to 100. 

During the system startup, we calculate the number of agents who have 
high conservation ethic barometer based on the highCEAgentsPercentage 
and number of land holders in the population.
\begin{equation}\label{xx}
\begin{split}
number of agents who have high CE = number of landholders * highCEAgentsPercentage
\end{split}
\end{equation}
Next, when initialising landholders, random values between 50 - 100 
are assigned for the agents who should have high conservation ethic 
barometer. Random values below 50 are assigned as conservation ethic 
barometers of other landholders. 

We do not control the number of agents who have high/low profit motive 
barometers. Therefore, random values within the range from 0 to 100 are 
assigned as profit motive barometers of agents. 


\subsection{Probability of participating}
Low probability set to 30\%, high probability set to 80\%. 

Probability of participation for LowC\_HighP category is calculated based 
on the number of successful bids in the last three auction rounds.
\begin{equation}\label{xx}
\begin{split}
probability = number of successful bids with a good profit/number of successful bids
\end{split}
\end{equation}
To calculate the \enquote{goodProfit}, the package, which is in the first quartile position
when all packages are sorted in the descending order of their opportunity
costs, is selected as the reference package. The \enquote{goodProfit} is the profit that 
can be achieved by the reference package when \enquote{highProfitPercentage} is applied on it.
\begin{equation}\label{xx}
\begin{split}
goodProfit = referencePackage.opportunityCost*(highProfitPercentage / 100)
\end{split}
\end{equation}
Eg: If there are 8 winning bids, and 2 of them make a good profit, there will be 25\% 
chance agents in this class will participate.

\subsection{Number of bids calculation}
We use two configurable variables to calculate number of bids.
\begin{itemize}
\item[-] defaultMaxNumberOfBids : ranges from 5 to 16.
\item[-] bidAddon : ranges from 5 to 10.
\end{itemize}
These two values are used to differentiate the number of bids 
selected by HighC\_LowP and HighC\_HighP categories.

\begin{tightitemize}
\item {\bf HighC\_LowP}\\ %many
Agents in this category select \enquote{defaultMaxNumberOfBids + 
bidAddon} number of packages.
\item {\bf HighC\_HighP}\\ %moderate
Agents in this category selects exactly 
\enquote{defaultMaxNumberOfBids} number of packages.
\item {\bf LowC\_HighP}\\  % all bids that give sufficient profit
No constraint on number of bids, but with current calculation will
tend to be about 20 of the 26 available packages.  \textit{Lin says:
  this seems quite a large number - get SS view on this}.
\textit{\\Sew says: It selects nearly 6 packages not 20.  
I had initially documented that reference package is the one at 
the 25th percentile. It should be corrected as 75th percentile.} 
\item {\bf LowC\_LowP}\\ %random
Random number between 1 and 26.

\end{tightitemize}

\subsection{Size of bids calculation}
Amount of bid is calculated to be in a low, medium or high range, by
taking an area above and below a low/medium/high threshold. The size
of the area above/below the threshold is configurable and is called
the proftVariability. The low threshold is always 0. The minimum for
the medium threshold is 10\% and the minimum for the high threshold is
20\%. The distaance between low and medium, and medium and high
threshholds is called the profitDifferential, and is configurable.
The profitDifferential ranges from 10-50\%. Possibly,
once we understand the effects of the size and level of separation of
these ranges we will simply use fixed ranges.
\begin{tightitemize}
\item {\bf HighC\_LowP}\\ % low profit range
A configurable range, centred around 0. Will likely decide a fixed
range centred around 0.
\item {\bf HighC\_HighP}\\ % medium range
A configurable range centred around a threshold that is a configurable
amount above 0. Will likely decide a fixed range around a fixed threshold.
\item {\bf LowC\_HighP}\\  % bids give good $ amount profit
To calculate what is a \enquote{good amount of profit}, we take the package
that is at the 75th percentile in terms of opportunity cost (i.e. \$
size of package) and apply the high profit percentage on it. This number is 
fixed for a given configuration with 100 rounds.

For a given round the agent uses a profit multiplier in the high
profit range (configurable around a high profit threshold, which is
the same distance from the medium threshold as that is from 0), and
submits all bids resulting in a profit above the 
\enquote{good amount of profit}. This will mostly be the packages including
and above the reference package in size, but with some variability
(due to using a profit range to determine bids).
\item {\bf LowC\_LowP}\\ % medium
Same as for HighC\_HighP -- medium profit range.
\end{tightitemize}

\subsection{Calculation of changes to conservation ethic based on results}
Following configurable parameters are used to update agents' conservation 
ethic barometer.
\begin{itemize}
\item[-] agentConservationEthicModifier : conservation ethic barometer is 
updated using this factor when the modification is done proportional to 
the profit. Ranges from 0.05 to 20.
\item[-] staticAgentConservationEthicModifier : a fixed value at 0.25. 
This is used to update conservation ethic barometer when static modifications 
to the barometer is needed.
\end{itemize}

\begin{tightitemize}
\item {\bf HighC successful}\\ % low-med profit down; high up; cost none.
The profit of the successful bid made by the agent is calculated as a percentage 
of the bid price. If the profit is within the range from zero to the upper 
margin of medium profit percentage range, decline agent's C proportional to 
the profit. 
\begin{equation}\label{xx}
\begin{split}
agent's new C = agent's current C ( 1- |profit| * agentConservationEthicModifier/100)
\end{split}
\end{equation}
If the agent's C is above the upper margin of medium profit percentage 
range, increase agent's C proportional to the profit.
\begin{equation}\label{xx}
\begin{split}
agent's new C = agent's current C ( 1+ |profit| * agentConservationEthicModifier/100)
\end{split}
\end{equation}

\item {\bf HighC unsuccessful}\\ % down prop to profit; cost small down
The highest profit made by any agent in the population is calculated as
 a percentage of the bid price. If the profit is below zero (no profit 
has been made), decline agent's C proportional to the profit. 
\begin{equation}\label{xx}
\begin{split}
agent's new C  = agent's current C * (1 - |profit| * agentConservationEthicModifier)
\end{split}
\end{equation}
If there is some profit, decline agents's C as below.
\begin{equation}\label{xx}
\begin{split}
agent's new C = agent's current C - staticAgentConservationEthicModifier
\end{split}
\end{equation}

\item {\bf HighC non-participant}\\  % as above
The highest profit made by any agent in the population is calculated 
as a percentage of the bid price. If the profit is above the medium 
profit percentage, decline agent's C proportional to the profit.
\begin{equation}\label{xx}
\begin{split}
agent's new C = agent's current  * (1 - |profit| * agentConservationEthicModifier/100)
\end{split}
\end{equation}

\item {\bf LowC successful }\\ % 0-med down; med-high up prop to profit
The profit of the successful bid made by the agent is calculated as a 
percentage of the bid price. If the profit is within the range from zero 
to the upper margin of medium profit percentage range, decline agent's 
C proportional to the profit. 
\begin{equation}\label{xx}
\begin{split}
agent's new C = agent's current C ( 1- |profit| * agentConservationEthicModifier/100)
\end{split}
\end{equation}
If the agent's C is above the upper margin of medium profit percentage 
range, increase agent's C proportional to the profit.
\begin{equation}\label{xx}
\begin{split}
agent's new C = agent's current C ( 1+ |profit| * agentConservationEthicModifier/100)
\end{split}
\end{equation}

\item {\bf LowC unsuccessful}\\ % small fixed down
Agent's C is always declined.
\begin{equation}\label{xx}
\begin{split}
agent's new C = agent's current C - staticAgentConservationEthicModifier
\end{split}
\end{equation}

\item {\bf LowC non-participant}\\ % small down prop to profit.
The highest profit made by any agent in the population is calculated as 
a percentage of the bid price. If the profit is above zero, decline agent's 
C proportional to the profit.
\begin{equation}\label{xx}
\begin{split}
agent's new C = agent's current  * (1 - |profit| * agentConservationEthicModifier/100)
\end{split}
\end{equation}

\end{tightitemize}

\subsection{Calculation of social norm increases in C}
The factor \enquote{socialNormUpdatePercentage} is used to update agents' 
C based on social norm. It ranges from 10 to 100.
If individual agent's C is below the average conservation ethic of all 
agents at the auction cycle start time, agent's C is increased using 
the socialNormUpdatePercentage. 
\begin{equation}
\begin{split}
new CE = current CE + (average CE - current CE) * socialNormUpdatePercentage/100
\end{split}
\end{equation}

\subsection{Calculation of changes to profit motive barometer}
\begin{tightitemize}
\item {\bf HichP}\\
The highest profit made by any agent in the population is calculated
 as a percentage of the bid price. Next, the agent's P is updated 
proportional to the profit using the factor agentProfitMotivationModifier. 
This factor ranges from 0.05 to 20.
\begin{equation}
\begin{split}
New P = Current P * (1 + |profit| * agentProfitMotivationModifier)
\end{split}
\end{equation}

\item {\bf LowP}\\
Agent's P increased by 0.25 (staticAgentProfitMotivationModifier)

\end{tightitemize}
\end{document}
